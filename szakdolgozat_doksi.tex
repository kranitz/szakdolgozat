\documentclass[a4paper,12pt,oneside]{report}

\usepackage{fancyhdr}
\usepackage[magyar]{babel}
\usepackage{t1enc}
\usepackage[utf8]{inputenc}
\usepackage{graphicx}
\usepackage{todonotes}
\usepackage[section,numbib,nottoc]{tocbibind}
\usepackage{hyperref}
\usepackage{amssymb}
\usepackage{booktabs}
\usepackage{pdflscape}
\usepackage{formai_kovetelmenyek}
\usepackage{pdfpages}
\usepackage{fixltx2e}
\usepackage{tikz}
\usepackage{floatrow}

\usepackage{array} % for defining a new column type
\usepackage{varwidth} %for the varwidth minipage environment


\hypersetup{
    pdfauthor={Kránitz Gábor},
    pdftitle={Tűzfalak tesztelése hálózati forgalom visszajátszással}
}

\lstset{
     basicstyle = \ttfamily\footnotesize
    ,breaklines = true
    ,prebreak   = \raisebox{0ex}[0ex][0ex]{\ensuremath{\hookleftarrow}}
    ,extendedchars = true
    ,literate={á}{{\'a}}1 {ó}{{\'o}}1 {é}{{\'e}}1 {í}{{\'i}}1 {ő}{{\~o}}1 {ö}{{\"o}}1 {ű}{{\'u}}1
}

\hyphenation{Google}

\title{Tűzfalak tesztelése hálózati forgalom visszajátszással}
\author{Kránitz Gábor}
\date{}

%fattyú- és árvasorok büntetése, ha nagyobb, akkor jobban próbálja elkerülni
\widowpenalty=400
\clubpenalty=400

\graphicspath{{./kepek/}}
\setcounter{secnumdepth}{3} %szamozza a subsubsection-oket is
\AtBeginDocument{\addtocontents{toc}{\protect\pagestyle{empty}}} %ezzel erem el, hogy a tartalomjegyzek ne kapjon oldalszamot
\AtBeginDocument{\addtocontents{tod}{\protect\thispagestyle{empty}}}


\lstset { %
    language=C++,
    backgroundcolor=\color{black!5}, % set backgroundcolor
    basicstyle=\footnotesize,% basic font setting
}


\begin{document}
\newcolumntype{M}{>{\begin{varwidth}{4cm}}l<{\end{varwidth}}} %M is for Maximal column

\setcounter{chapter}{1}

\pagestyle{empty}
%------------------------------------------------------------------
% külsõ kötéstábla
{
    \begin{center}
    \vspace*{5cm}
    {
        \Huge SZAKDOLGOZAT}\\
        \vspace*{10cm}
        {\LARGE Kránitz Gábor}\\
        \vspace*{3cm}
        {\LARGE 2015}
    \end{center}
}
\newpage

% címoldal
\begin{center}
{
    \Large Pannon Egyetem\\
    Villamosmérnöki és Információs Rendszerek Tanszék\vspace*{3mm}\\
    Programtervező Informatikus BSc szak
}
    \vspace*{2cm}\\
    {\LARGE \bf SZAKDOLGOZAT}
    \vspace{3cm}\\
    {\LARGE\bf Tűzfalak tesztelése hálózati forgalom visszajátszással }
    \vspace{3cm}\\
    {\large Kránitz Gábor}
    \vspace{6cm}
    \\
    {\large Témavezető: Dulai Tibor}\\
    {\large Külső konzulens: Tollas Ferenc}
    \vspace{1cm}\\
    {\large 2014}
\end{center}
\normalsize
% címlap vége
\newpage

Ide jön az eredeti vagy a fénymásolt feladatkiírás.
\newpage

\begin{center}
\section*{Nyilatkozat}
\end{center}

Alulírott Kránitz Gábor diplomázó hallgató kijelentem, hogy a szakdolgozatot a Pannon Egyetem Villamosmérnöki és Információs Rendszerek tanszéken készítettem Programtervező informatikus BSc szak (BSc in Computer Science
) megszerzése érdekében.

Kijelentem, hogy a szakdolgozatban lévő érdemi rész saját munkám eredménye, az érdemi részen kívül csak a hivatkozott forrásokat (szakirodalom, eszközök, stb.) használtam fel.

Tudomásul veszem, hogy a szakdolgozatban foglalt eredményeket a Pannon Egyetem, valamint a feladatot kiíró szervezeti egység saját céljaira szabadon felhasználhatja.\\

\begin{flushleft}
{Veszprém, 2015. december 02.\\}
\end{flushleft}

\begin{flushright}
{Aláírás \vspace{4cm}}
\end{flushright}

Alulírott Dulai Tibor témavezető kijelentem, hogy a szakdolgozatot Kránitz Gábor a Pannon Egyetem Villamosmérnöki és Információs Rendszerek tanszéken készítette Programtervező informatikus BSc szak (BSc in Computer Science
) megszerzése érdekében.

Kijelentem, hogy a szakdolgozat védésre bocsátását engedélyezem.\\

\begin{flushleft}
{Veszprém, 2015. december 02.\\}
\end{flushleft}

\begin{flushright}
{Aláírás}
\end{flushright}
%A tartalomjegyzék:
\newpage
\pagebreak
\begin{center}
\section*{Köszönetnyilvánítás}
\end{center}

Köszönöm a családomnak a sok türelmet és segítséget, amit kaptam, nélkülük ez a szakdolgozat nem készült volna el.
\\
\\
Köszönöm témavezetőmnek, Dulai Tibornak az elmúlt egy év során adott iránymutatását.
\\
\\
Végül, de nem utolsó sorban, szeretném megköszönni munkatársaimnak a sok segítséget, szaktársaimnak a bíztatást.

\newpage

\begin{center}
\section*{\textbf{\Large \MakeUppercase{Tartalmi összefoglaló}}}
\end{center}

tartalmi osszefoglalo

\vspace{2cm}

{\bf Kulcsszavak:} {\it Zorp, TcpForwarder}
\newpage

\newpage

\begin{center}
\section*{\textbf{\Large \MakeUppercase{Abstract}}}
\end{center}

english sum

\vspace{2cm}

{\bf Keywords:} {\it Zorp, TcpForwarder}
\newpage
%--------------%------------------------------------------------------------------
\pagenumbering{gobble} %ne legyen oldalszamozas a tartalomjegyzek oldalon
%\listoftodos

\renewcommand{\thefigure}{\arabic{figure}}


\setcounter{tocdepth}{3} %subsubsection-ok is latszodjanak
\thispagestyle{empty}
\tableofcontents
\pagebreak

\pagenumbering{arabic} %legyen oldalszamozas
\setcounter{page}{1} %innentől indul az oldalszámozás
\pagestyle{plain}
\fancyhead[C]{\rightmark}
\fancyfoot[R]{\thepage}

\section{A feladat összefoglalása}


\subsection{Első lépések}


\subsubsection{Képfeldolgozási képességek}
\begin{itemize}
	\itemsep0em
	\item Képjavító eljárások, pl.: élesítés, kontrasztkiegyenlítés
	\item Geometriai műveletek, pl.: átméretezés, forgatás, tükrözés
	\item Analizálás, pl.: eltérések detektálása, alacsony szintű képleírók
	\item Szerkesztési műveletek, pl.: logikai, szöveg, alakzatok elhelyezése
	\item Színterek közötti konverzió, pl.: RGB $\rightarrow $ HSL, csatornák külön kezelése stb

\end{itemize}

\subsection{Választott szoftverek}

\end{document}